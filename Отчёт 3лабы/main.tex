\documentclass{article}
\usepackage[utf8]{inputenc}
\usepackage{cmap}					% поиск в PDF
\usepackage{mathtext} 				% русские буквы в формулах
\usepackage[T2A]{fontenc}			% кодировка
\usepackage[utf8]{inputenc}			% кодировка исходного текста
\usepackage[english,russian]{babel}	% локализация и переносы
\usepackage{indentfirst}

\title{Отчёт по 3 лабе, содержание проекта}
\date{October 2022}

\begin{document}

\maketitle

\section{Структура проекта}
\section{Описание использованных команд}
\section{История изменения файлов и работы с проектом}

\newpage
\huge {Структура проекта} \\ \\
\large{Проект содержит 2 ветки - main(основная) и резервная(Brovkin)}
\\ \\
\large{В основной ветке содержатся описание проекта и его версии}
\\ \\
\large{В резервной ветке хранится README с описанием проекта и первая лабораторная (Автоматизация сборки проекта)}
\\ \\ \\ \\
1. config
\\ \\
В данном файле содержатся настройки Git репозитория. Например, здесь можно хранить email и имя пользователя.
\\ \\
2. description
\\ \\
Данный файл предназначен для GitWeb и содержит в себе информацию о проекте (название проекта и его описание). GitWeb - это веб интерфейс, написанный для просмотра Git репозитория используя веб-браузер. Если вы не пользуетесь GitWeb, то это не столь важно.
\\ \\
3. hooks
\\ \\
В этом каталоге Git предоставляет набор скриптов, которые могут автоматически запускаться во время выполнения git команд. В некоторых случаях это значительно упрощает разработку. Например, вы можете написать скрипт, который будет редактировать сообщение коммита согласно вашим требованиям.
\\ \\
4. info - exclude
\\ \\
Каталог info содержит файл exclude, в котором можно указывать любые файлы, и Git не станет добавлять их в свою историю. Это почти то же самое что и .gitingnore (возможно вы сталкивались с ним. Его можно найти в корневом каталоге вашего проекта), за тем исключением, что exclude не сохраняется в истории проекта, и вы не сможете им поделиться с другими.
\\ \\
5. refs
\\ \\
Каталог refs хранит в себе копию ссылок на объекты коммитов в локальных и удаленных ветках.
\\ \\
6. logs
\\ \\
Каталог logs хранит в себе историю проекта для всех веток в вашем проекте.
\\ \\
7. objects
\\ \\
Каталог objects хранит в себе BLOB объекты, каждый из которых проиндексирован уникальным SHA.
\\ \\
8. index
\\ \\
Промежуточная область с метаданными, такими как временные метки, имена файлов, а также SHA файлов, которые уже упакованы Git. В эту область попадают файлы, над которыми вы работали, при выполнение команды git add.
\\ \\
9. HEAD
\\
Файл содержит ссылку на текущую ветку, в которой вы работаете
\\ \\
10. ORIGHEAD
\\ \\
Каждый раз во время слияния в этот файл попадает SHA ветки, с которой проводилось слияние
\\ \\
11. FETCHHEAD
\\ \\
Файл хранит в себе ссылки в виде SHA на ветки, которые участвовали в git fetch
\\ \\
12. MERGEHEAD
\\ \\
Файл хранит в себе ссылки в виде SHA на ветки, которые участвовали в git merge
\\ \\
13. COMMITEDITMSG
\\ \\
Файл содержит в себе последнее введенное вами сообщение коммита

\newpage
\huge {Описание использованных команд} 
\\ \\
\large{Для создания ветки github`а и её обновлением использовался github desktop}
\\ \\
\Large{Команды которые были выполнены с помощью github desktop}
\\ \\
\large{Команды которые были использованы}
\\ \\
\large{Добавить изменения в индекс(staging area) - git add}
\\ \\
\large{Закоммитить изменения - git commit}
\\ \\
\large{Запушить - git push}
\\ \\
\large{Команда git status отображает состояние директории и индекса(staging area). Это позволяет определить, какие файлы в проекте отслеживаются Git, а также какие изменения будут включены в следующий коммит.}
\\ \\
\large{git remote add origin - Эта команда сопоставит удаленное хранилище с ссылкой на локальный репозиторий. С этого момента можно обращаться к удаленному репозиторию через  ссылку}
\\ \\
\large{git push origin - отправление результатов нашей работы в репозиторий.}
\\ \\ \large{git branch "'что-то"' - Создание ветки в проекте}
\\ \\ \large{git checkout "'что-то"' - Переключение на другую ветку проекта}
\\ \\ \\ \\
\Large{Команды которые были прописаны \\ работа с историей}
\\ \\
\large{git log - Данная команда предназначена для отображения всей нашей истории. Она может быть весьма удобна, если нам понадобилось узнать, какие изменения мы вносили ранее. Или если нам нужно откатиться до определенного места в истории, либо если есть нужда её отредактировать.}
\\ \\
\large{Команда git show используется для отображения полной информации о любом объекте в Git, будь то коммит или ветка. По умолчанию git show отображает информацию коммита, на который в данный момент времени указывает HEAD.}
\\ \\
\large{git reflog - Эта команда выводит упорядоченный список коммитов, на которые указывал HEAD. Грубо говоря, она отображает историю всех ваших перемещений по проекту. Основное преимущество этой команды заключается в том, что если мы вдруг случайно удалили часть истории или откатились назад, мы сможем проинспектировать момент утраты нужной вам информации и откатиться обратно.}
\\ \\ \\ \\
\Large{Команды которые не использовались, но могут пригодиться при работе с историей}
\\ \\ \\ \\
\large{git reset -  Она позволяет откатить проект до определенной точки. Эту команду можно использовать с тремя параметрами:}
\\ \\ \large{1. В случае с --soft, содержимое вашего индекса, а также рабочей директории, остается неизменным. Это значит, что если мы откатимся назад на пару коммитов, мы изменим ссылку указателя HEAD на указанный коммит и все изменения, которые были до этого внесены, окажутся в индексе.}
\\ \\ \large{2. При использовании параметра --mixed, мы опять-таки изменим ссылку указателя HEAD, но все предыдущие изменения в индекс не попадут, а будут отслеживаться как не занесенные в индекс. Это дает возможность внести в индекс только те изменения, которые нам необходимы, что довольно удобно!}
\\ \\ \large{3. Если использовать команду git reset с параметром --hard, мы снова изменим ссылку указателя HEAD, но все предыдущие изменения не попадут ни в индекс, ни в зону отслеживаемых файлов. Это значит, что мы полностью сотрем все изменения, которые вносили ранее. Это также удобно, если вы знаете, что вам больше не пригодится ваша предыдущая работа над проектом.}
\\ \\ \large{git branch - Посмотреть текущую ветку проекта}
\\ \\ \large{git merge dev - Слияние двух веток воедино}

\newpage

\huge{История изменения файлов и работы с проектом}
\\ \\ \Large{История всех коммитов}
\\ \\ \large{commit 0241876b5ab45f93e98a2450f8a9e6f336d135ea (HEAD -> Brovkin, origin/Brovkin) \\ \\
Author: BrovkinArtem <artembro2014@mail.ru> \\
Date:   Mon Nov 14 03:29:18 2022 +0300
\\ \\
    \textbf{Delete - удаление файла питона из проекта (был лишним)}
\\ \\
commit 051b8e094b80a77ea89e0ffe7414c3e46e300915 \\
Author: BrovkinArtem <artembro2014@mail.ru> \\
Date:   Mon Nov 14 03:28:25 2022 +0300 \\
\\ \\
    \textbf{Update - добавление README файла в папку src 1 лабораторной}
\\ \\
commit e86b5efaec0ededb7744cc9669eb223c7aec251a \\
Author: BrovkinArtem <artembro2014@mail.ru> \\ 
Date:   Mon Nov 14 02:41:06 2022 +0300 \\
\\ \\
    \textbf{Modify - Обновление main.c файла (добавление дополнительного описания)}
\\ \\
commit 81d1b9805b0494c3f7388982cd277733373dbb89 \\
Author: BrovkinArtem <artembro2014@mail.ru> \\
Date:   Mon Nov 14 02:14:16 2022 +0300 \\
\\ \\
    \textbf{add - Добавление папки первой лабораторной в резервную ветку}
\\ \\
commit 9ec63325e51777b2d49712d2fb57dd06ea6992b5 \\
Author: BrovkinArtem <artembro2014@mail.ru> \\
Date:   Mon Nov 14 02:03:36 2022 +0300 \\
\\ \\
    \textbf{first - Загрузка резервного проекта Casetools в репозиторий гитхаба netToolsLab3}
\\ \\
commit a50364470ce8636bbbcdcaee4597bef072c4d1e1 \\
Author: BrovkinArtem <78733604+BrovkinArtem@users.noreply.github.com> \\
Date:   Mon Nov 14 02:02:17 2022 +0300 \\
\\ \\
    \textbf{Initial commit - проинициализирование Git директории в своем проекте и добавление README файла} \\
(END)}
\\ \\ \large{}
\end{document}
