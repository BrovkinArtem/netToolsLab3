\documentclass{article}
\usepackage[utf8]{inputenc}
\usepackage{cmap}					% поиск в PDF
\usepackage{mathtext} 				% русские буквы в формулах
\usepackage[T2A]{fontenc}			% кодировка
\usepackage[utf8]{inputenc}			% кодировка исходного текста
\usepackage[english,russian]{babel}	% локализация и переносы
\usepackage{indentfirst}
\usepackage{graphicx}
\graphicspath{ {./images/} }

\title{Отчёт по 5 лабе}
\date{October 2022}

\begin{document}

\maketitle

\section{Программа}
\section{Выбранная программа или сайт для форматирования стиля и выбранные настройки}
\section{Сравнение исходного кода и отформатированного}

\newpage
\huge {Программа} \\ \\
\Large{Для форматирования была выбрана программа cookiesСount.php для подсчета количества посещений страницы удаленным пользователем с использованием Cookies. \\ (Предполагается, что прием Cookies не запрещен браузером.) \\ При первом посещении сценарий выводит строку \\ \textbf{"Добро пожаловать!"}. \\ При N-ом посещении выводится строка \\ \textbf{"Вы посетили эту страницу N раз"}}
\\ \\

\includegraphics[scale=0.7]{безФорматирования.png}

исходная программа
 \\ \\
\Large{Программа специально написана без пробелов и прочего чтобы показать работу программы для форматирования}

\newpage
\huge{Выбранная программа или сайт для \\ форматирования стиля и \\ выбранные настройки} \\ \\
\Large{Для реализации форматирования кода был выбран сайт phpformatter}
\\ \\
\Large{Выбранные настройки}\\
\begin{itemize}
    \item  Стиль отступа - Pear style
    \item  Отступы - табом
    \item  Сколько табов использовать - 4
    \item  Отступ регистра и значение переключателя - общий
    \item  Правилно выровненные операторы присваивания - вкл
    \item  Пробелы между структурой управления и скобкой в if - вкл
    \item  Пробелы между математическими операторами умножения, деления, сложения и вычитания - вкл
    \item  Пробелы между логическими операторами - вкл
    \item  Пробелы между операторами равенства - вкл
    \item  Пробелы между операторами отношений - вкл
    \item  Пробелы между побитовыми операторами - вкл
    \item  Пробелы между операторами присваивания - вкл
    \item  Пробелы между операторами указания (=>) - вкл
\end{itemize}
\newpage
\huge{Сравнение исходного кода и отформатированного} \\ \\
\includegraphics[scale=0.7]{безФорматирования.png}

\large{исходная программа}
\\ \\
\includegraphics[scale=0.8]{сФорматированием.png}

\large{отформатированная программа}


\end{document}
