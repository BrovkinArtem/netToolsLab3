\documentclass{article}
\usepackage[utf8]{inputenc}
\usepackage{cmap}					% поиск в PDF
\usepackage{mathtext} 				% русские буквы в формулах
\usepackage[T2A]{fontenc}			% кодировка
\usepackage[utf8]{inputenc}			% кодировка исходного текста
\usepackage[english,russian]{babel}	% локализация и переносы
\usepackage{indentfirst}
\usepackage{graphicx}
\graphicspath{ {./images/} }

\title{Отчёт по 6 лабе}
\date{October 2022}

\begin{document}

\maketitle

\section{Программа}
\section{найденные ошибок}
\section{Методы исправления}

\newpage
\huge {Программа} \\ \\
\large{Splint, сокращение от Secure Programming Lint, представляет собой инструмент программирования для статической проверки программ на C на наличие уязвимостей и ошибок в кодировании. Ранее называвшийся LCLint, он представляет собой современную версию Unix lint tool.
\\ \\
Splint обладает способностью интерпретировать специальные примечания к исходному коду, что обеспечивает более строгую проверку, чем это возможно, просто просматривая только исходный код. Splint используется gpsd как часть усилий по проектированию с нулевыми дефектами.
\\ \\
Splint - это свободное программное обеспечение, выпущенное на условиях GNU General Public License.
\\ \\
Основная деятельность по разработке Splint прекратилась в 2010 году. Согласно резюме на SourceForge, по состоянию на сентябрь 2012 года последнее изменение в репозитории было в ноябре 2010 года. Репозиторий Git на GitHub имеет более свежие изменения, начиная с июля 2019 года.}

\newpage
\huge{найденные ошибок} \\ \\
\large{Ошибки искались в программе для сортировки чисел}
\\
\includegraphics[scale=1.3]{Screenshot_2.png}

исходная программа
\\ \\
\newpage
\Large{Всего ошибок 4}
\begin{itemize}
    \item Возвращаемое значение (тип int) игнорируется в: scanf(''\%i'', \&n)
    \item Возвращаемое значение (тип int) игнорируется: scanf(''\%i'', \&a[i])
    \item Значение a[], используемое перед определением.  Используется значение rvalue, которое не может быть инициализировано значением на некотором пути выполнения.
    \item Значение a[], используемое перед определением. (Уже в другом месте программы)
\end{itemize}

\newpage
\huge{Методы исправления} \\ \\
\begin{itemize}
    \item Если это предназначается исплользование в функции, то можно привести результат к (void), чтобы исключить сообщение.
    \item Если это предназначается исплользование в функции, то можно привести результат к (void), чтобы исключить сообщение.
    \item приемлемым обходным путем является просто очистка памяти, чтобы splint обрабатывал ее как инициализированную, иначе никак не фиксится
    \item приемлемым обходным путем является просто очистка памяти, чтобы splint обрабатывал ее как инициализированную, иначе никак не фиксится
\end{itemize}

\end{document}
